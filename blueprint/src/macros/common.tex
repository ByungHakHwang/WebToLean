% In this file you should put all LaTeX macros to be used
% both by the pdf version and the web version.
% This should be most of your macros.



\newtheorem{thm}{Theorem}[section]
\newtheorem{prop}[thm]{Proposition}
\newtheorem{lem}[thm]{Lemma}
\newtheorem{cor}[thm]{Corollary}
\newtheorem{conj}[thm]{Conjecture}
\theoremstyle{definition}
\newtheorem{defn}[thm]{Definition}
\newtheorem{exam}[thm]{Example}
\newtheorem{fact}[thm]{Fact}
\newtheorem{ques}[thm]{Question}
\newtheorem{prob}[thm]{Problem}
\newtheorem{rmk}[thm]{Remark}
\theoremstyle{remark}


\newcommand\CC{\mathbb{C}}
\newcommand\ZZ{\mathbb{Z}}
\newcommand\Spec{\mathcal{S}}

\newcommand\SYT{\operatorname{SYT}}
\newcommand\Dyck{\operatorname{Dyck}}
\newcommand\NC{\operatorname{NC}}
\newcommand\NN{\operatorname{NN}}
\newcommand\Mat{\operatorname{Mat}}
\newcommand\Cr{\operatorname{Cr}}
\newcommand\SL{\operatorname{SL}}
\newcommand\AC{\operatorname{AC}}
\newcommand\rlmin{\operatorname{rlmin}}
\newcommand\cc{\operatorname{c}}
\newcommand\Web{\operatorname{Web}}

\newcommand\NS{\mathsf{N}}
\newcommand\ES{\mathsf{E}}

\newcommand\OnlyXmarking[2]
{\draw[very thick] (#1-0.7,#2-0.7)--(#1-0.3,#2-0.3);
\draw[very thick] (#1-0.7,#2-0.3)--(#1-0.3,#2-0.7); }
\newcommand\Xmarking[2]
{\draw[very thick] (#1-0.7,#2-0.7)--(#1-0.3,#2-0.3);
\draw[very thick] (#1-0.7,#2-0.3)--(#1-0.3,#2-0.7);
\draw (#1-1,#2-0.5)--(#1-0.5,#2-0.5);
\draw (#1-0.5,#2-0.5)--(#1-0.5,#2); }
\newcommand\Cross[2]
{\draw (#1-0.5,#2)--(#1-0.5,#2-1);
\draw (#1,#2-0.5)--(#1-1,#2-0.5); }
\newcommand\UP[2]{\draw (#1-0.5,#2)--(#1-0.5,#2-1);}
\newcommand\EAST[2]{\draw (#1,#2-0.5)--(#1-1,#2-0.5);}
\newcommand\Asmooth[2]
{\draw (#1,#2-0.5) .. controls (#1-0.45,#2-0.45) and (#1-0.45,#2-0.45) .. (#1-0.5,#2);
\draw (#1-1,#2-0.5) .. controls (#1-0.55,#2-0.55) and (#1-0.55,#2-0.55) .. (#1-0.5,#2-1); }
\newcommand\Bsmooth[2]
{\draw (#1-1,#2-0.5) .. controls (#1-0.55,#2-0.45) and (#1-0.55,#2-0.45) .. (#1-0.5,#2);
\draw (#1,#2-0.5) .. controls (#1-0.45,#2-0.55) and (#1-0.45,#2-0.55) .. (#1-0.5,#2-1); }
\newcommand\Matching[2]{\draw (#1,0) to [out=55,in=125] (#2,0);}

\newcommand\SYM{\mathfrak{S}}

\newcommand\qand{\quad\mbox{and}\quad}
